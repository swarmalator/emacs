% Created 2019-10-31 Thu 16:54
% Intended LaTeX compiler: pdflatex
\documentclass[11pt]{article}
\usepackage[utf8]{inputenc}
\usepackage[T1]{fontenc}
\usepackage{graphicx}
\usepackage{grffile}
\usepackage{longtable}
\usepackage{wrapfig}
\usepackage{rotating}
\usepackage[normalem]{ulem}
\usepackage{amsmath}
\usepackage{textcomp}
\usepackage{amssymb}
\usepackage{capt-of}
\usepackage{hyperref}
\author{Cole Turner}
\date{\today}
\title{}
\hypersetup{
 pdfauthor={Cole Turner},
 pdftitle={},
 pdfkeywords={},
 pdfsubject={},
 pdfcreator={Emacs 26.3 (Org mode 9.2.5)}, 
 pdflang={English}}
\begin{document}

\tableofcontents

This script will allow you to run commands on the X-MW Controller.

In order to do this you need to make sure the command server is enabled
\begin{enumerate}
\item You will need to make sure that you can use SSH
\begin{itemize}
\item on windows you can install an SSH client like PuTTY
\item on Linux/Mac you can use the built in SSH command
\end{itemize}
\item connect the X-MW Controller to the same network as your computer via an ethernet cable
\item find the ip address of the Controller
\begin{itemize}
\item click the Home icon in the top right
(if you just started the Controller it should already show the Home screen)
\item click the Question icon in the top right
\item if you connected it directly to your computer use the "192.168.10.2"
otherwise use the other ipv4 address
\end{itemize}
\item Login via SSH at that address with username: "pi", password: "raspberry"
example: "ssh pi@192.168.1.146"
\item You should see a prompt like this "pi@raspberrypi:\textasciitilde{} \$"
\item Edit the X-MW.ini file
\begin{itemize}
\item type "nano \textasciitilde{}/.X-MW.ini" and press ENTER
\item go to the line that says "host\textsubscript{cmd} = default" and change it to "host\textsubscript{cmd} = true"
\item save and exit the file by pressing CTRL+X then "Y" to the "Save modified buffer" prompt
\item then press ENTER when it asks for the filename to save it into
\end{itemize}
\item Reboot the controller by typing "sudo reboot" then ENTER
\end{enumerate}

Now the command server is enabled

to use the command server follow these steps
\begin{enumerate}
\item make sure the Controller is connected to the network like in the previous steps
\item You will need Python 3.5 or later installed on your computer
(most Mac and Linux systems already have this)
\begin{itemize}
\item on windows you can get Python 3.5 or later by going to "\url{https://python.org/download}"
\end{itemize}
\item run this script by opening a terminal in the directory where you downloaded the code
and typing "python ./X\textsubscript{MWcontroller}-interactive.py" when the script prompts you for the hostname,
enter the ip address
make sure you run this in python 3 (on Linux/Mac the command is "python3") on windows
you may need to find the python3 executable and run that passing script as an argument
\item you should now be in an interactive prompt. you can type commands here and press ENTER
to send them to the Controller.
\item to see a list of all the commands or to learn how to use the commands type "help"
\begin{itemize}
\item now you type commands
\end{itemize}
\end{enumerate}

You may also use the python script as a module and import it into other scripts.
the only variable you need to import is class XMWController.
To see how to create a python script that runs these commands similar
\end{document}
